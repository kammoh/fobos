\chapter{Using the FOBOS Profiler} \label{chap:fobos-profiler}
FOBOS profiler can be used to map time domain events or clock cycles to the specific sample in a power trace (or t-value).
This can be use in applications like:
\begin{itemize}
 \item Determine when leakage on a t-test occurs (clock cycle).
 \item Determining instruction power consumption.
 \item Correlating the state of a state machine to specific power trace samples.
\end{itemize}

\section{FOBOS trigger settings}
Once the DUT starts the crypto operation, it sets \texttt{di\_ready} to zero. 
The Control Board can trigger the oscilloscope immediately when \texttt{di\_ready=0} or after this event by any number of clock cycles. 
The trigger signal can be active for any number of clock cycles or until the DUT is done with its operation.
The falling and rising edges of the trigger signal can be used later to truncated the trace properly to be used by the profiler.
\newline 
optionally, a user may generate a file that maps the internal state of the DUT to clock cycles. This is called a 'state file'. 
This file can be generated by modifying the test bench to write a value corresponding to each state to a text file.
The profiler script can take this state file as input along with the power traces (or t-values) and display clock transitions and the DUT state on each clock cycle.



\section{Generating State Files}

The following is an example on how to modify the fobos\_dut\_tb.vhd to write the state of the DUT to a text file.
Writing starts just after \texett{di\_ready} becomes zero until \texttt{do\_valid} goes to one. That is the time when the DUT is processing the data.
\newline
In fobos\_dut\_tb.vhd, we add this process:

\begin{verbatim}
--Process to write states 
writeState: PROCESS(clk)
  VARIABLE VectorLine: LINE;
BEGIN
IF (rising_edge(CLK)) THEN
        IF (di_ready = '0' and do_valid = '0') THEN
          hwrite(VectorLine, state_debug);  
          writeline(stateFile, VectorLine);	 
        END IF;
END IF;
ASSERT False
Report "Writing States"
SEVERITY NOTE;
END Process;
--------
\end{verbatim}

We also add these lines in architecture declarations section:

\begin{verbatim}
---
FILE stateFile: TEXT OPEN WRITE_MODE is "state_file.txt";
signal state_debug : std_logic_vector(7 downto 0);
---
\end{verbatim}

In this example, state is assumed to be 8 bit std\_logic\_vector reported by the DUT (victim) controller (i.e FOBOS\_DUT have a port called \texttt{state\_debug})
The file generated is a list of numbers each representing the state of the controller at the clock cycle identified by the line number.

\section{Profiler configuration}

Below is an example profiler configuration with description of each parameter.
This configuration is located at \texttt{fobos/config/analysis.ini}
\begin{verbatim}
[profiler]
#File to read the t-t_values (input)
#Expected at the measurements directory
srcFile     = t_values.npy
#Enable/Disable clock plotting
display_clk     = YES 
#num of clocks to in trace
num_of_clks     = 10  
#high clock voltage for plotting
clk_high        = 5   
#low clock voltage for plotting
clk_low         = -5  
#File name to save plot
profilerPlot    = profiler_plot.png
#File that stors clock number vs state mapping
stateFile       = state_file.txt
\end{verbatim}

\section{Running the profiler}

Run the \texttt{fobos/bin/runProfiler.py} script.
This will display a menu showing all the measurements done in the current project.
You can select a measurement and the profiler will take it as input. A new analysis directory will be created under the selected project directory to store output files.



\section{Using the plot\_t\_values.py script}

The \texttt{plot\_t\_values.py} is a script that can be used to run the profiler on t-values. Here is usage description.\newline
Command line arguments: \newline

\begin{verbatim}
$ python plot_t_values.py -h
usage: plot_t_values.py [-h] t_values plot_file

positional arguments:
  t_values    A .npy file that store traces as Nx1 Nupmy array that contains
              t-values.
  plot_file   File name where the plot is saved

optional arguments:
  -h, --help  show this help message and exit
\end{verbatim}
  
Also there are settings on the script that need to be set. Here is a snippet of the configuration along with brief explanation
for the configuration parameters:
\begin{verbatim}
##########GENERAL SETTINGS                                                                                                                                                                                                                   
start_ylim  = -40  #Plot ymlim range                                                                                                                                                                                                         
end_ylim    = 40                                                                                                                                                                                                                             
##########END GENERAL SETTINGS                                                                                                                                                                                                               
###########CLK GRAPH SETTINGS                                                                                                                                                                                                                
display_clk = 'YES' #Enable/Disable clock plotting                                                                                                                                                                                           
num_of_clks = 18   #num of clocks to in trace. Known from behavioral simulation.                                                                                                                                                                                              
clk_high    = 10   #high clock amplitude for plotting                                                                                                                                                                                          
clk_low     = -10  #low clock amplitude for plotting                                                                                                                                                                                           
state_file  = 'state_map.txt' #text file that includes states. One state in each line.                                                                                                                                                                
###END CLK GRAPH SETTING
\end{verbatim}
