\chapter{Data Acquisition} \label{chap:dataAcquisition}

After test vectors have been generated, user can run dataAcquisition.py. The PC will send one test vector at a time to the control board, which sends it to DUT. 
The control board will trigger the oscilloscope to capture the power trace. The process will be repeated unitl all traces are collected.
 

The DUT Wrapper uses the header information in the test vector to put the data (plaintext, key etc.) into the correct FIFOs. 
The DUT Wrapper then allows the DUT (victim) algorithm to run by setting the victim reset to zero. The victim then drains the FIFOs (sdi,pdi and rdi FIFOs) and stores the output in the dout FIFO. 
Once the dout FIFO accumulates the expected amount of data, the DUT wrapper sends data to the controller which sends it to the PC.

Figure~\ref{fig:fobos-capture} shows the components of FOBOS including the handshake signals used.
Before running the \texttt{dataAcquisstion.py} script, the  user must  modify the configuration files at \texttt{config/config.txt} and \texttt{confi/acquisitionconfig.txt}.
Here is sample for \texttt{acquisitionConfig.txt} file.

\section{Data Aquisition Configuration}

\subsection{General Settings}
\begin{itemize}
 \item MEASUREMENT\_FORMAT \newline
 Possible values: dat \newline
 The format to store power measurements. Only dat is supported.
 \item LOGGING \newline 
 Possible values: INFO$|$DEBUG \newline
 The logging level. DEBUG is more verbose.
\end{itemize}

\subsection{Control Board Settings}
\begin{itemize}
 \item CONTROL\_BOARD \newline
 Possible values: Nexys3$|$Nexys2 \newline
 The Control Board type. 
 \item VICTIM\_RESET \newline
 Possible values: Integer \newline
 Reset the DUT after the specified clock cycles.
 \item TIME\_OUT \newline
 Possible values: Integer \newline
 If the DUT will not return data after the specified clock cycles, a time-out signal is returned to the PC.
\end{itemize}


\subsection{Trigger Settings}

The Control Board can send a trigger to the Oscilloscope once the DUT starts processing the data (ie. di\_ready = 0). Or it can be configured to trigger any number of clock cycles after this event occurs.
\begin{itemize}
 \item TRIGGER\_WAIT\_CYCLES \newline
 The number of clock cycles after which the trigger is asserted (after di\_ready goes to zero).
 \item TRIGGER\_LENGTH\_CYCLES \newline
 The time the trigger signal is asserted.
 \item TRIGGER\_TYPE \newline 
 possible values: TRG\_NORM, TRG\_FULL, TRG\_NORM\_CLK, TRG\_FULL\_CLK
        \begin{itemize}
          \item TRG\_NORM \newline Normal trigger mode. in this mode the TRIGGER\_WAIT\_CYCLES and TRIGGER\_LENGTH\_CYCLES are applied.
          \item TRG\_FULL \newline Full trigger mode. While DUT is running (between di\_ready = 0 and do\_valid = 1) the trigger is asserted.
          \item TRG\_NORM\_CLK \newline Similar to TRG\_NORM but the trigger signal is anded with the clock.
          \item TRG\_FULL\_CLK \newline Similar to TRG\_FULL but the trigger signal is anded with the clock.
       \end{itemize}

 \item CUT\_MODE \newline Controls how the trace retreived from the scope will be processed.
        possible values: FULL, TRIG\_HIGH
        \begin{itemize}
          \item FULL \newline The trace is cut starting at the rising edge of the trigger to the end of the screen.
          \item TRIG\_HIGH \newline the trace is cut from the rising edge to the falling edge of the trigger ie. the trace where the trigger is high will be saved.
        \end{itemize}
\end{itemize}  

\subsection{Data Settings}
\begin{itemize}
 \item DATA\_FILE \newline
 Possible values: file name (string) \newline
 The test-vector file name.
 \item EXPECTED\_OUTPUT \newline
 Possible values: Integer \newline
 Expected output size in bytes.
 \item OUTPUT\_FORMAT \newline
 Possible values: hex \newline
 Output format. 'hex' is the only supported format.
\end{itemize}

\subsection{Capture Settings}
\begin{itemize}
  \item NUMBER\_OF\_TRACES \newline
 Number of traces to collect.
 \item CAPTURE\_MODE  \newline
 Possible values: MULTI$|$SINGLE \newline
 Sigle encryption per traces or mutiple encryptions per trace.
\end{itemize}

\subsection{Oscilloscope Settings}
\begin{itemize}
 \item OSCILLOSCOPE \newline
 Possible values: AGILENT$|$OPENADC \newline
 Oscilloscope type.
 \item OSCILLOSCOPE\_IP \newline
 Possible values: IP address \newline
 Oscilloscope IP address.
 \item OSCILLOSCOPE\_PORT \newline
 Oscilloscope port number.
 \item AUTOSCALE = YES$|$NO \newline
 
 \item IMPEDANCE \newline
 Possible values: FIFTY$|$ONEMEG \newline
 Oscilloscope input impedance.
 \item CHANNELx\_RANGE  \newline
 Possible values: ON$|$OFF$|$voltage range \newline
 Oscilloscope channel voltage range (Full screen range) in Volts.
 \item TIME\_RANGE \newline
 Possible values: float (seconds)
 Oscilloscope time range in seconds (full screen time).
 \item TIMEBASE\_REF = LEFT    \newline
 \item TRIGGER\_THRESHOLD \newline
 Minimum voltage for valid trigger.
 \item TRIGGER\_SOURCE \newline
 Possible values: CHANNEL1$|$CHANNEL2$|$CHANNEL3$|$CHANNEL4 \newline
 Specifies the rigger channel.
 \item TRIGGER\_MODE =  EDGE
 \item TRIGGER\_SWEEP = NORM
 \item TRIGGER\_LEVEL = 1
 \item TRIGGER\_SLOPE = POSITIVE
 \item ACQUIRE\_TYPE = NORM$|$PEAK$|$HRES$|$AVER
 \item ACQUIRE\_MODE =  RTIM$|$ETIM$|$SEG

\end{itemize}

\section{Sample Configuration File}

\begin{verbatim}
 # ============================================= 
# Global Settings 
# ============================================= 
# ============================================= 
MEASUREMENT_FORMAT = dat # Default => dat 
LOGGING = INFO # INFO|DEBUG 
# ============================================= 
# ============================================= 
# Control Board Settings 
# ============================================= 
# ============================================= 
CONTROL_BOARD = Nexys3 
TRIGGER_WAIT_CYCLES = 0 #@VICTIM CLOCK 
TRIGGER_LENGTH_CYCLES = 1 #@VICTIM CLOCK 
TRIGGER_TYPE = TRG_FULL #TRG_NORM | TRG_FULL | TRG_NORM_CLK | TRG_FULL_CLK 
CUT_MODE = TRIG_HIGH #FULL | TRIG_HIGH 
# ============================================= 
# ============================================= 
# Test Data Generation Settings 
# ============================================= 
# ============================================= 
DATA_FILE     = dinFile.txt 
EXPECTED_OUTPUT = 16 # Expected output size in bytes 
OUTPUT_FORMAT = hex # Default => hex 
NUMBER_OF_ENCRYPTIONS_PER_TRACE = 1 
BLOCK_SIZE = 16 # In Bytes 
# ============================================= 
# ============================================= 
# FOBOS Capture Settings 
# ============================================= 
# ============================================= 
DUMMY_RUN = NO #YES/NO 
NUMBER_OF_TRACES = 50000 
#################################################### 
######## Signal Alignment Module Parameters ######## 
#################################################### 
CAPTURE_MODE = SINGLE # MULTI|SINGLE 
TRIGGER_THRESHOLD = 1.0 
# ============================================= 
# ============================================= 
# FOBOS Oscilloscope Settings 
# ============================================= 
# ============================================= 
# INTIALIZATION OPTIONS 
OSCILLOSCOPE = AGILENT #AGILENT|OPENADC 
OSCILLOSCOPE_IP = 192.168.10.10 
OSCILLOSCOPE_PORT = 5025 
AUTOSCALE = NO # YES|NO
IMPEDANCE = ONEMEG #FIFTY|ONEMEG 
# VOLTAGE AND TIME RANGE OPTIONS
CHANNEL1_RANGE = 0.060V 
CHANNEL2_RANGE = 6V 
CHANNEL3_RANGE = OFF # ON|OFF|voltage range 
CHANNEL4_RANGE = OFF # ON|OFF|voltage range 
TIME_RANGE = 0.000050 
TIMEBASE_REF = LEFT
# TRIGGER OPTIONS 
TRIGGER_SOURCE = CHANNEL2 
TRIGGER_MODE = EDGE
TRIGGER_SWEEP = NORM 
TRIGGER_LEVEL = 1 
TRIGGER_SLOPE = POSITIVE 
# ACQUIRE OPTIONS 
ACQUIRE_TYPE = NORM # NORM|PEAK|HRES|AVER 
ACQUIRE_MODE = RTIM # RTIM | ETIM| SEG
\end{verbatim}

\section{Running Data Acquisition}
Once the configuration is done, user can run 

\texttt{python dataAcquisition.py }

The output will be saved in \texttt{workspace/$<$projectName$>$/measurements}. The traces are stored in a Numpy array called \texttt{rawDataAligned.npy}.




