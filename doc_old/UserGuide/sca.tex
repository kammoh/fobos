\chapter{Side Channel Analysis}
\section{Introduction}% to Side-Channel Analysis}
Recent years have seen a dramatic increase of market adoption and utility of so called "smart" devices
by people from all walks of life. These devices play a central role in how people are entertained, communicate,
network, work, bank and shop. Yet for every positive outcome from these devices, there is often a corollary risk.
For example, let us consider a smart phone. On one hand, there are billions of applications which provide 
unprecedented ease of access to a plethora of applications or simply termed \emph{apps} to meet any user requirements. 
On the other hands, they are also are providing a fertile environment for the distribution of hostile apps or malware.
Also, the increased power of these smart phones makes them more suitable for a host of business purposes, which can also result
in the exposure and compromise of corporate data and systems. Finally, the very portability of mobile devices means that
they are highly susceptible to loss and theft. Thus there is great need in protecting information accessed by these devices
and this information is usually secured using cryptographic algorithms. 

According to Kerchoff's Law (or Shannon's Maxim)~\cite{klaws}, \newline
\parbox[c]{\textwidth}{\textit{a cryptosystem's security must be solely based on the 
		secret key even if everything about the underlying encryption algorithm is public knowledge.}} 
However, physical implementations in hardware as well as in software 
of such encryption algorithms have been shown to
leak secret information in the form of so called side-channels
and also during sudden change in operational characteristics of the crypto-device 
i.e. via \emph{Fault Injection}. The side-channel leakage could be in the 
form of power consumption~\cite{694}, electro magnetic radiation~\cite{811} or timing~\cite{694} 
of the device. The side-channels leak sensitive information whenever the device performs an 
operation using the secret data. Attacks which make use of such inhAnalysiserent physical leakage are called 
side-channel attacks \Gls{SCA}. \Gls{SCA} is a new research area of applied cryptanalysis that has
gained popularity since mid nineties. The research in this area shows that \Gls{SCA} pose a major 
threat because the physical implementations of the cryptographic
devices are difficult to control and often result in unintended leakage of information.
Generally, all hardware implementations of cryptographic algorithms are assumed to be vulnerable to side 
channel cryptanalysis, if there are no special precautions in the implementation.

\section{Power Analysis}
   \subsection{Simple Power Analysis (SPA)}
   \subsection{Differential/Correlation Power Analysis (DPA/CPA)}
       \subsubsection{Difference of Means}
        \subsubsection{Spearman Rank Coefficient}
        \subsubsection{Pearson's r}
      
   \subsection{Power Model}
%      \begin{itemize}
     \subsubsection{Hamming Distance (HD)}
     Hamming Distance \Gls{HD}
      \subsubsection{Hamming Weight (HW)}
      Hamming Weight \Gls{HW}
 %     \end{itemize}