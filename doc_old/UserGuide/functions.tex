\chapter{Function Descriptions}

\section{FOBOS - Analysis Module}
FOBOS's analysis module uses a set of python scripts to post process the raw measurement data obtained from the oscilloscope and perform analysis on the obtained data.
Various functions implemented in the Analysis module is described below:\newline

\begin{table}
\caption{getAlignedMeasuredPowerData Function}
\begin{tabular}{ |p{2cm}||p{11cm}|  }
 \hline
 \multicolumn{2}{|c|}{\textbf{signalAnalysisModule.getAlignedMeasuredPowerData}} \\
 \hline
 Usage & \texttt{signalAnalysisModule.getAlignedMeasuredPowerData()}\\ \hline
 Inputs & None \\ \hline
 Outputs & M x N Numpy matrix that holds aligned traces. Where M is the number of traces and N is the number of samples per trace.\\ \hline
 Description & Reads aligned traces from the \$Workspace/\$projectName/\$attempt/Measurements directory and loads it to an M x N Numpy matrix.
This function calls signalAnalysisModule.readAlignedDataFromFile(). \\ \hline
\end{tabular}
\end{table}

\begin{table}
\caption{readAlignedDataFromFile Function}
\begin{tabular}{ |p{2cm}||p{11cm}|  }
 \hline
 \multicolumn{2}{|c|}{\textbf{signalAnalysisModule.readAlignedDataFromFile}} \\
 \hline
 Usage & \texttt{signalAnalysisModule.readAlignedDataFromFile()}\\ \hline
 Inputs & None \\ \hline
 Outputs & M x N Numpy matrix that holds aligned traces. Where M is the number of traces and N is number of samples per trace.\\ \hline
 Description & Reads aligned data from file and returns M x N Numpy matrix. \\ \hline
\end{tabular}
\end{table}

\begin{table}
\caption{detectSampleSize Function}
\begin{tabular}{ |p{2cm}||p{11cm}|  }
 \hline
 \multicolumn{2}{|c|}{\textbf{signalAnalysisModule.detectSampleSize}} \\
 \hline
 Usage & \texttt{signalAnalysisModule.detectSampleSize(fileName)}\\ \hline
 Inputs & Aligned traces file name. \\ \hline
 Outputs & Number of samples in a trace. \\ \hline
 Description & Reads the first 10 traces and returns the number of samples in the largest trace.
If the number of traces is less than 10 all traces are read.
This is done to be able to adjust all traces to the same number of samples if some do not have the same size due to acquisition timing. \\ \hline
\end{tabular}
\end{table}

\begin{table}
\caption{adjustSampleSize Function}
\begin{tabular}{ |p{2cm}||p{11cm}|  }
 \hline
 \multicolumn{2}{|c|}{\textbf{signalAnalysisModule.adjustSampleSize}} \\
 \hline
 Usage & \texttt{signalAnalysisModule.adjustSampleSize(sampleLength, dataArray)}\\ \hline
 Inputs & \begin{itemize} \item sample length to adjust to. 		   
         \item N x 1 nympy array that represents one trace where N is the number of samples in the trace. 
		\end{itemize}\\ \hline
 Outputs & SampleLenght x 1 Numpy array that represents the adjusted trace. \\ \hline
 Description & Used to modify the number of samples in a trace. If the number of samples is less than SampleLength, the array is padded with zeros. If the number of samples is more than SampleLength, the array is truncated. The function does nothing if the number of samples is equal to SampleLength.
 \\ \hline
\end{tabular}
\end{table}

\begin{table}
\caption{acquirePowerModel Function}
\begin{tabular}{ |p{2cm}||p{11cm}|  }
 \hline
 \multicolumn{2}{|c|}{\textbf{signalAnalysisModule.acquirePowerModel}} \\
 \hline
 Usage & \texttt{signalAnalysisModule.acquirePowerModel
 (HyptheticalDataFileName,globals.ADAPTIVE\_CPA)}\\ \hline
 Inputs & \begin{itemize}
 			\item Power model file name
 			\item  Correlation type 
 			\end{itemize}
 			\\ \hline
 Outputs & M x N Numpy array that holds the hypothetical power traces. Where N is the number of encryptions/decryptions and M is the number of key guesses. \\ \hline
 Description & Reads the hypothetical power data from file in \$fobos/data. Returns M x N Numpy array that holds the hypothetical power traces. Where N is the number of encryptions/decryptions and M is the number of key guesses.
 \\ \hline
\end{tabular}
\end{table}


\begin{table}
\caption{computeAlignedData Function}
\begin{tabular}{ |p{2cm}||p{11cm}|  }
 \hline
 \multicolumn{2}{|c|}{\textbf{signalAnalysisModule.computeAlignedData}} \\
 \hline
 Usage & \texttt{signalAnalysisModule.computeAlignedData(powerData, triggerData)}\\ \hline
 Inputs &  \begin{itemize}
 		    \item N x 1 Numpy array that holds measured power data. Where N is the number of samples.
 		    \item N x 1 Numpy array that holds trigger power data. Where N is the number of samples.
 		    \end{itemize} \\ \hline
 Outputs & Aligned power trace (K x 1 Numpy array where K is the number of samples). \\ \hline
 Description & Uses powerData and triggerData to generate the aligned trace.
The function looks for the rising edge of the trigger signal to determine the start of the trace. \\ \hline
\end{tabular}
\end{table}

\begin{table}
\caption{correlation\_pearson Function}
\begin{tabular}{ |p{2cm}||p{11cm}|  }
 \hline
 \multicolumn{2}{|c|}{\textbf{sca.correlation\_pearson}} \\
 \hline
 Usage & \texttt{sca.correlation\_pearson(measuredPowerData, hypotheticalPowerData)}\\ \hline
 Inputs &  \begin{itemize}
 		    \item M x N Numpy matrix for power traces. Where M is the number of encryptions/decryptions, N the number of samples per trace.
            \item L x M array the represents the hypothetical power values. Where M is the number of encryptions/decryptions and L is the number of key guesses (i.e for byte guess, L=256).
            \end{itemize}\\ \hline
 Outputs & N x L Numpy correlation matrix where N is the number of samples per trace. \\ \hline
 Description & Calculates Pearson correlation between the hypothetical data and measured data. Returns an N x L correlation matrix.
When we guess byte values L = 256.
 \\ \hline
\end{tabular}
\end{table}


\begin{table}
\caption{findMinimumGuessingEntropy Function}
\begin{tabular}{ |p{2cm}||p{11cm}|  }
 \hline
 \multicolumn{2}{|c|}{\textbf{sca.findMinimumGuessingEntropy}} \\
 \hline
 Usage & \texttt{signalAnalysisModule.computeAlignedData(measuredPowerData, triggerData)}\\ \hline
 Inputs &  \begin{itemize}
 		   \item measured power data.
 		   \item hypotheticalPowerData.
 		   \end{itemize}\\ \hline
 Outputs &  Minimum Guessing Entropy graph \\ \hline
 Description &  calculates the find Minimum Guessing Entropy \\ \hline
\end{tabular}
\end{table}



\begin{table}
\caption{plotTrace Function}
\begin{tabular}{ |p{2cm}||p{11cm}|  }
 \hline
 \multicolumn{2}{|c|}{\textbf{plottingModule.plotTrace}} \\
 \hline
 Usage & \texttt{plottingModule.plotTrace(dataToPlot, traceNos, plotType)}\\ \hline
 Inputs &  \begin{itemize}
 		   \item M x N Numpy matrix that holds traces to plot. Where M is the number of traces and N is the number of samples.
 		   \item Trace number
 		   \item Plot type
 		   \end{itemize}
per trace.
The traces to be plotted.
  \\ \hline
 Outputs & None \\ \hline
 Description & Plots the traces represented by the Numpy matrix. The x-axis represents time and the y-axis represents voltage. \\ \hline
\end{tabular}
\end{table}

\begin{table}
\caption{plotHist Function}
\begin{tabular}{ |p{2cm}||p{11cm}|  }
 \hline
 \multicolumn{2}{|c|}{\textbf{plottingModule.plotHist}} \\
 \hline
 Usage & \texttt{plottingModule.plotHist(corrMatrix, corrType)}\\ \hline
 Inputs &  \begin{itemize}
 		   \item M x N Correlation Numpy matrix. 
 		   \item Correlation type.
 		   \end{itemize}  \\ \hline
 Outputs & None \\ \hline
 Description & Plots a histogram. The x-axis represents the key guess and the y-axis represenst the number of occurrences.
 \\ \hline
\end{tabular}
\end{table}

\begin{table}
\caption{plotCorr Function}
\begin{tabular}{ |p{2cm}||p{11cm}|  }
 \hline
 \multicolumn{2}{|c|}{\textbf{plottingModule.plotCorr}} \\
 \hline
 Usage & \texttt{plottingModule.plotCorr(corrMatrix, corrType)}\\ \hline
 Inputs & \begin{itemize}
 		  \item M x N Numpy correlation matrix
 		  \item Correlation type 
 		  \end{itemize} \\ \hline
 Outputs & None \\ \hline
 Description & Plots correlation data. \\ \hline
\end{tabular}
\end{table}

\begin{table}
\caption{sampleSpaceDisp Function}
\begin{tabular}{ |p{2cm}||p{11cm}|  }
 \hline
 \multicolumn{2}{|c|}{\textbf{postProcessingModule.sampleSpaceDisp}} \\
 \hline
 Usage & \texttt{postProcessingModule.sampleSpaceDisp(alignedData)}\\ \hline
 Inputs & M x N Numpy matrix that holds aligned data. Where M is the number of traces and N is the number of samples per trace. \\ \hline
 Outputs & M x Window\_Size Numpy array that holds the aligned data after removing samples  \\ \hline
 Description & Removes samples before WINDOW\_START and after WINDOW\_START + WINDOW\_SIZE - 1 from each trace. \\ \hline
\end{tabular}
\end{table}

\begin{table}
\caption{compressData Function}
\begin{tabular}{ |p{2cm}||p{11cm}|  }
 \hline
 \multicolumn{2}{|c|}{\textbf{compressData}} \\
 \hline
 Usage & \texttt{compressData(measuredPowerData)}\\ \hline
 Inputs & M x N Numpy array that represents traces. Where M is the number of traces and N is the number of samples per trace. \\ \hline
 Outputs & M x K Numpy array that represents compressed traces. Where M is the number of traces and $K = \frac{N}{COMPRESSION\_LENGTH}$.  \\ \hline
 Description & Summarizes COMPRESSION\_LENGTH samples into one sample. The summarization type depends on the COMPRESSION\_TYPE configuration parameter which can be MEAN, MIN or Max. \\ \hline
\end{tabular}
\end{table}
\begin{table}
\caption{compress Function}
\begin{tabular}{ |p{2cm}||p{11cm}|  }
 \hline
 \multicolumn{2}{|c|}{\textbf{postProcessingModule.compress}} \\
 \hline
 Usage & \texttt{postProcessingModule.compress (a, compressionLenght, compressionType)}\\ \hline
 Inputs & \begin{itemize}
 		  \item M x N Numpy array that represents traces. Where M is the number of traces and N is the number of samples per trace.
 		  \item Number of samples to compress into one sample.
          \item Compression type.
          \end{itemize} \\ \hline
 Outputs &  M x K Numpy array that represents compressed traces. Where M is the number of traces and $K = \frac{N}{COMPRESSION\_LENGTH}$.  \\ \hline
 Description & This function is called by postProcessingModule.compressData() to do the compression. \\ \hline
\end{tabular}
\end{table}

\begin{table}
\caption{traceExpunge Function}
\begin{tabular}{ |p{2cm}||p{11cm}|  }
 \hline
 \multicolumn{2}{|c|}{\textbf{postProcessingModule.traceExpunge}} \\
 \hline
 Usage & \texttt{postProcessingModule.traceExpunge(measuredPowerData)}\\ \hline
 Inputs & M x N Numpy array that represents traces. Where M is the number of traces and N is the number of samples per trace. \\ \hline
 Outputs & L x N Numpy array that represents traces after removing the traces that do not fall in acceptable range. Where L is the number of traces and N is the number of samples per trace.  \\ \hline
 Description & Removes traces that do not fall in acceptable range. \\ \hline
\end{tabular}
\end{table}


\section{FOBOS - Capture Module}

The Capture module is used to run encryptions on hardware and capture traces from the ocsilloscope.

\begin{table}
\caption{openOscilloscopeConnection Function}
\begin{tabular}{ |p{2cm}||p{11cm}|  }
 \hline
 \multicolumn{2}{|c|}{\textbf{Oscilloscope\_core.openOscilloscopeConnection}} \\
 \hline
 Usage & \texttt{Oscilloscope\_core.openOscilloscopeConnection()}\\ \hline
 Inputs & None  \\ \hline
 Outputs &  None \\ \hline
 Description & Connects to oscilloscope. It opens a socket using the IP address OSCILLOSCOPE\_IP and port number OSCILLOSCPOE\_PORT. Also gets the oscilloscope identifier. \\ \hline
\end{tabular}
\end{table}

\begin{table}
\caption{setOscilloscopeConfigAttributes Function}
\begin{tabular}{ |p{2cm}||p{11cm}|  }
 \hline
 \multicolumn{2}{|c|}{\textbf{Oscilloscope\_core.setOscilloscopeConfigAttributes}} \\
 \hline
 Usage & \texttt{Oscilloscope\_core.setOscilloscopeConfigAttributes()}\\ \hline
 Inputs & None  \\ \hline
 Outputs &  None \\ \hline
 Description & Configures the oscilloscope by sending commands (in text format) to the oscilloscope. \\ \hline
\end{tabular}
\end{table}

\begin{table}
\caption{initializeOscilloscopeDataStorage Function}
\begin{tabular}{ |p{2cm}||p{11cm}|  }
 \hline
 \multicolumn{2}{|c|}{\textbf{Oscilloscope\_core.initializeOscilloscopeDataStorage}} \\
 \hline
 Usage & \texttt{Oscilloscope\_core.initializeOscilloscopeDataStorage()}\\ \hline
 Inputs & None \\ \hline
 Outputs &  None \\ \hline
 Description & Creates empty Numpy arrays for each enabled oscilloscope channel. \\ \hline
\end{tabular}
\end{table}

\begin{table}
\caption{armOscilloscope Function}
\begin{tabular}{ |p{2cm}||p{11cm}|  }
 \hline
 \multicolumn{2}{|c|}{\textbf{Oscilloscope\_core armOscilloscope}} \\
 \hline
 Usage & \texttt{Oscilloscope\_core.armOscilloscope()}\\ \hline
 Inputs & None \\ \hline
 Outputs &  None \\ \hline
 Description & Instructs the oscilloscope to digitize channels specified in FOBOS configuration. \\ \hline
\end{tabular}
\end{table}

\begin{table}
\caption{populateOscilloscopeDataStorageAndAlign Function}
\begin{tabular}{ |p{2cm}||p{11cm}|  }
 \hline
 \multicolumn{2}{|c|}{\textbf{Oscilloscope\_core.populateOscilloscopeDataStorageAndAlign}} \\
 \hline
 Usage & \texttt{Oscilloscope\_core.populateOscilloscopeDataStorageAndAlign(traceCount)}\\ \hline
 Inputs & The number of current trace. \\ \hline
 Outputs &  None \\ \hline
 Description & Reads power data trace from oscilloscope and trigger signal trace. It then aligns the trace to the trigger signal and saves the aligned trace to file. \\ \hline
\end{tabular}
\end{table}


\begin{table}
\caption{closeOscilloscopeConnection Function}
\begin{tabular}{ |p{2cm}||p{11cm}|  }
 \hline
 \multicolumn{2}{|c|}{\textbf{Oscilloscope\_core.closeOscilloscopeConnection}} \\
 \hline
 Usage & \texttt{Oscilloscope\_core.closeOscilloscopeConnection()}\\ \hline
 Inputs & None \\ \hline
 Outputs &  None \\ \hline
 Description & Closes socket that connects to oscilloscope. \\ \hline
\end{tabular}
\end{table}

\begin{table}
\caption{openControlBoardConnection Function}
\begin{tabular}{ |p{2cm}||p{11cm}|  }
 \hline
 \multicolumn{2}{|c|}{\textbf{Oscilloscope\_core.openControlBoardConnection}} \\
 \hline
 Usage & \texttt{usbcomm\_core.openControlBoardConnection()}\\ \hline
 Inputs & None \\ \hline
 Outputs &  None \\ \hline
 Description & Initializes connection to control board, resets control board and reads control board and victim clocks. \\ \hline
\end{tabular}
\end{table}

\begin{table}
\caption{initializeControlBoardConnection Function}
\begin{tabular}{ |p{2cm}||p{11cm}|  }
 \hline
 \multicolumn{2}{|c|}{\textbf{usbcomm\_core.initializeControlBoardConnection}} \\
 \hline
 Usage & \texttt{usbcomm\_core.initializeControlBoardConnection()}\\ \hline
 Inputs & None \\ \hline
 Outputs &  None \\ \hline
 Description & Initializes the USB connection to the board.
Called from OpenControlBoardConnection(). \\ \hline
\end{tabular}
\end{table}


\begin{table}
\caption{sendTriggerParamsToControlBoard Function}
\begin{tabular}{ |p{2cm}||p{11cm}|  }
 \hline
 \multicolumn{2}{|c|}{\textbf{usbcomm\_core.sendTriggerParamsToControlBoard}} \\
 \hline
 Usage & \texttt{usbcomm\_core.sendTriggerParamsToControlBoard()}\\ \hline
 Inputs & None \\ \hline
 Outputs &  None \\ \hline
 Description & Sends the trigger parameters to the control boards. Parameters are: TRIGGER\_WAIT\_CYCLES and TRIGGER\_LENGTH\_CYCLES. \\ \hline
\end{tabular}
\end{table}

\begin{table}
\caption{runEncrytionOnControlBoard Function}
\begin{tabular}{ |p{2cm}||p{11cm}|  }
 \hline
 \multicolumn{2}{|c|}{\textbf{usbcomm\_core.runEncrytionOnControlBoard}} \\
 \hline
 Usage & \texttt{usbcomm\_core.runEncrytionOnControlBoard(traceCount)}\\ \hline
 Inputs & The number of block used in encryption. \\ \hline
 Outputs &  None \\ \hline
 Description & Sends a block of data to control board to do encryption. 
 The key is sent before sending the frist block. \\ \hline
\end{tabular}
\end{table}


\begin{table}
\caption{sendKeyToControlBoard Function}
\begin{tabular}{ |p{2cm}||p{11cm}|  }
 \hline
 \multicolumn{2}{|c|}{\textbf{usbcomm\_core.sendKeyToControlBoard()}} \\
 \hline
 Usage & \texttt{usbcomm\_core.sendKeyToControlBoard()}\\ \hline
 Inputs & None \\ \hline
 Outputs &  None \\ \hline
 Description & Sends the key to control board. 
 This function is called from usbcomm\_core.runEncrytionOnControlBoard().\\ \hline
\end{tabular}
\end{table}

\begin{table}
\caption{sendBlockOfDataToControlBoard Function}
\begin{tabular}{ |p{2cm}||p{11cm}|  }
 \hline
 \multicolumn{2}{|c|}{\textbf{usbcomm\_core.sendBlockOfDataToControlBoard}} \\
 \hline
 Usage & \texttt{usbcomm\_core.usbcomm\_core.sendBlockOfDataToControlBoard
 (traceCount)}\\ \hline
 Inputs & The number of block used in encryption \\ \hline
 Outputs &  None \\ \hline
 Description & Sends a block of data to the control board. 
 This function is called from usbcomm\_core.runEncrytionOnControlBoard(). \\ \hline
\end{tabular}
\end{table}

\begin{table}
\caption{saveControlBoardOutputDataStorage Function}
\begin{tabular}{ |p{2cm}||p{11cm}|  }
 \hline
 \multicolumn{2}{|c|}{\textbf{usbcomm\_core\_core.saveControlBoardOutputDataStorage}} \\
 \hline
 Usage & \texttt{Oscilloscope\_core.saveControlBoardOutputDataStorage()}\\ \hline
 Inputs & None \\ \hline
 Outputs &  None \\ \hline
 Description & Saves output from control board (cipher text) to file. File is stored in \$Workspace/\$projectName/\$attempt/output/ \\ \hline
\end{tabular}
\end{table}


\section{FOBOS - Other Functions}

Here we list helper functions that are used by the Analysis and Capture modules.

\begin{table}
\caption{getKeyForAnalysis Function}
\begin{tabular}{ |p{2cm}||p{11cm}|  }
 \hline
 \multicolumn{2}{|c|}{\textbf{dataGenerator.getKeyForAnalysis}} \\
 \hline
 Usage & \texttt{dataGenerator.getKeyForAnalysis()}\\ \hline
 Inputs & None \\ \hline
 Outputs & Key formated as a list of hexadecimal bytes.  \\ \hline
 Description & Reads the key from file in \$Workspace/\$project/\$attempt/output/ directory. \\ \hline
\end{tabular}
\end{table}

\begin{table}
\caption{getPlainText Function}
\begin{tabular}{ |p{2cm}||p{11cm}|  }
 \hline
 \multicolumn{2}{|c|}{\textbf{dataGenerator.getPlainText}} \\
 \hline
 Usage & \texttt{dataGenerator.getPlainText()}\\ \hline
 Inputs & None \\ \hline
 Outputs & A list of blocks that represents plain text.  \\ \hline
 Description & Generates random plain text or reads from file depending on configuration.
 Plain text file is located in \$fobos/\$sources/ directory. \\ \hline
\end{tabular}
\end{table}

\begin{table}
\caption{generateRandomKey Function}
\begin{tabular}{ |p{2cm}||p{11cm}|  }
 \hline
 \multicolumn{2}{|c|}{\textbf{dataGenerator.generateRandomKey}} \\
 \hline
 Usage & \texttt{dataGenerator.generateRandomKey()}\\ \hline
 Inputs & None \\ \hline
 Outputs & A list of key bytes. Each byte is represented as a hexadecimal string. \\ \hline
 Description & Generates a random key in hexadecimal format. Key size is read form the KEY\_SIZE configuration parameter. \\ \hline
\end{tabular}
\end{table}

\begin{table}
\caption{convertToHex Function}
\begin{tabular}{ |p{2cm}||p{11cm}|  }
 \hline
 \multicolumn{2}{|c|}{\textbf{dataGenerator.convertToHex}} \\
 \hline
 Usage & \texttt{dataGenerator.convertToHex(hexString)}\\ \hline
 Inputs & Hexadecimal string  \\ \hline
 Outputs &  \\ \hline
 Description &  \\ \hline
\end{tabular}
\end{table}
\begin{table}
\caption{configureWorkspace Function}
\begin{tabular}{ |p{2cm}||p{11cm}|  }
 \hline
 \multicolumn{2}{|c|}{\textbf{configExtract.configureWorkspace()}} \\
 \hline
 Usage & \texttt{configExtract.configureWorkspace()}\\ \hline
 Inputs & None \\ \hline
 Outputs &  None \\ \hline
 Description & Creates the project directory in the workspace and creates directories to store measured power data, cipher text and plain text etc.
 It also copies some configuration files and other files into the project directory. \\ \hline
\end{tabular}
\end{table}

\begin{table}
\caption{extractConfigAttributes Function}
\begin{tabular}{ |p{2cm}||p{11cm}|  }
 \hline
 \multicolumn{2}{|c|}{\textbf{configExtract.extractConfigAttributes()}} \\
 \hline
 Usage & \texttt{configExtract.extractConfigAttributes()}\\ \hline
 Inputs & None \\ \hline
 Outputs &  None \\ \hline
 Description & Reads the main configuration file to get configuration attributes. It also reads the acquisition configuration file and extracts configuration attributes. \\ \hline
\end{tabular}
\end{table}

\begin{table}
\caption{updatePowerAndTriggerFileNames Function}
\begin{tabular}{ |p{2cm}||p{11cm}|  }
 \hline
 \multicolumn{2}{|c|}{\textbf{configExtract.updatePowerAndTriggerFileNames}} \\
 \hline
 Usage & \texttt{configExtract.updatePowerAndTriggerFileNames()}\\ \hline
 Inputs & None \\ \hline
 Outputs &  None \\ \hline
 Description & Checks for the existance of measured data files and trigger data file and sets variables to the file names.  \\ \hline
\end{tabular}
\end{table}

\begin{table}
\caption{configureAnalysisWorkspace Function}
\begin{tabular}{ |p{2cm}||p{11cm}|  }
 \hline
 \multicolumn{2}{|c|}{\textbf{configExtract.configureAnalysisWorkspace}} \\
 \hline
 Usage & \texttt{configExtract.configureAnalysisWorkspace()}\\ \hline
 Inputs & None \\ \hline
 Outputs &  None \\ \hline
 Description & Configures the analysis workspace directory by creating directories to store analysis results and copies configuration files. \\ \hline
\end{tabular}
\end{table}

\begin{table}
\caption{extractAnalysisConfigAttributes Function}
\begin{tabular}{ |p{2cm}||p{11cm}|  }
 \hline
 \multicolumn{2}{|c|}{\textbf{configExtract.extractAnalysisConfigAttributes}} \\
 \hline
 Usage & \texttt{configExtract.extractAnalysisConfigAttributes(fileName)}\\ \hline
 Inputs & Configuration file name. \\ \hline
 Outputs &  None \\ \hline
 Description & Reads the file provided and gets configuration attributes. Also, copies the file to the project’s local configuration directory for future reference. \\ \hline
\end{tabular}
\end{table}

\begin{table}
\caption{goToSleep Function}
\begin{tabular}{ |p{2cm}||p{11cm}|  }
 \hline
 \multicolumn{2}{|c|}{\textbf{support.goToSleep}} \\
 \hline
 Usage & \texttt{support.goToSleep(value)}\\ \hline
 Inputs & Time to sleep in seconds. \\ \hline
 Outputs &  None \\ \hline
 Description & Sleep for number of seconds. \\ \hline
\end{tabular}
\end{table}

\begin{table}
\caption{exitProgram Function}
\begin{tabular}{ |p{2cm}||p{11cm}|  }
 \hline
 \multicolumn{2}{|c|}{\textbf{support.exitProgram}} \\
 \hline
 Usage & \texttt{support.exitProgram()}\\ \hline
 Inputs & None \\ \hline
 Outputs &  None \\ \hline
 Description & Self-explanatory \\ \hline
\end{tabular}
\end{table}

\begin{table}
\caption{wait Function}
\begin{tabular}{ |p{2cm}||p{11cm}|  }
 \hline
 \multicolumn{2}{|c|}{\textbf{support.wait}} \\
 \hline
 Usage & \texttt{support.wait()}\\ \hline
 Inputs & None \\ \hline
 Outputs &  None \\ \hline
 Description & Waits for the user to press Enter to continue program execution. \\ \hline
\end{tabular}
\end{table}


\begin{table}
\caption{clear\_screen Function}
\begin{tabular}{ |p{2cm}||p{11cm}|  }
 \hline
 \multicolumn{2}{|c|}{\textbf{support.clear\_screen}} \\
 \hline
 Usage & \texttt{support.clear\_screen()}\\ \hline
 Inputs & None \\ \hline
 Outputs &  None \\ \hline
 Description & Self-explanatory. \\ \hline
\end{tabular}
\end{table}


\begin{table}
\caption{convertToByteArray Function}
\begin{tabular}{ |p{2cm}||p{11cm}|  }
 \hline
 \multicolumn{2}{|c|}{\textbf{support.convertToByteArray}} \\
 \hline
 Usage & \texttt{support.convertToByteArray(hexString)}\\ \hline
 Inputs & Hexadecimal string \\ \hline
 Outputs &  A byte array \\ \hline
 Description & Converts a hexadecimal string to a byte array. \\ \hline
\end{tabular}
\end{table}


\begin{table}
\caption{arrayToString Function}
\begin{tabular}{ |p{2cm}||p{11cm}|  }
 \hline
 \multicolumn{2}{|c|}{\textbf{support.arrayToString(array)}} \\
 \hline
 Usage & \texttt{support.arrayToString(array)}\\ \hline
 Inputs & Array to convert \\ \hline
 Outputs &  A string that consist of array elements \\ \hline
 Description & Self-explanatory. \\ \hline
\end{tabular}
\end{table}


\begin{table}
\caption{readFile Function}
\begin{tabular}{ |p{2cm}||p{11cm}|  }
 \hline
 \multicolumn{2}{|c|}{\textbf{support.readFile}} \\
 \hline
 Usage & \texttt{support.readFile(fileName)}\\ \hline
 Inputs & File name \\ \hline
 Outputs &  A string that holds file content.  \\ \hline
 Description & Self-explanatory. \\ \hline
\end{tabular}
\end{table}



\begin{table}
\caption{removeFile Function}
\begin{tabular}{ |p{2cm}||p{11cm}|  }
 \hline
 \multicolumn{2}{|c|}{\textbf{Support.removeFile}} \\
 \hline
 Usage & \texttt{Support.removeFile(fileName)} \\ \hline
 Inputs & File name \\ \hline
 Outputs &  None  \\ \hline
 Description & Self-explanatory. \\ \hline
\end{tabular}
\end{table}

\begin{table}
\caption{removeComments Function}
\begin{tabular}{ |p{2cm}||p{11cm}|  }
 \hline
 \multicolumn{2}{|c|}{\textbf{Support.removeComments(datalist)}} \\
 \hline
 Usage & \texttt{Support.removeComments(datalist)}\\ \hline
 Inputs & Data list \\ \hline
 Outputs &   \\ \hline
 Description & Removes comments (anything after a '\#' sign) from a list of strings. \\ \hline
\end{tabular}
\end{table}

 
