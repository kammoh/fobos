 \chapter{Software Installation}

FOBOS consists of Python scripts for acquisition and analysis and hardware for controller and DUT.
This chapter describes installation of the software part (the Python scripts).
Since Python is used, the software can run in Linux, Windows and other operating systems.

The following installation procedure was tested on Linux Ubuntu 16.04. It is assumed that Pytho2.7 is already installed.
\begin{enumerate}
\item Download FOBOS from \url{https://cryptography.gmu.edu/fobos/getfobos.php}
\item Extract the archive into the directory of your choice

\texttt{\$ tar xvfz fobos-v1.0.tgz }

\item Use the following commands to install the necessary Python packages:

\texttt{\$ sudo apt-get install python-pip} \newline
\texttt{\$ sudo pip install pycrypto} \newline
\texttt{\$ sudo pip install numpy} \newline
\texttt{\$ sudo pip install matplotlib} \newline
\texttt{\$ sudo apt-get install python-tk} \newline
\texttt{\$ sudo pip install scipy} \newline
\texttt{\$ sudo pip install configparser} \newline
\item Install Digilent Adept from
\url{https://reference.digilentinc.com/reference/software/adept/start?redirect=1#software_downloads}
This will install libraries needed for the Digilent DEP protocol used for PC - Control Board communication.
\item Install Xilinx ISE. Make sure to install Cable Drivers in Xilinx ISE installation. If not installed, you will encounter issues using Impact to program the boards.
\end{enumerate}

  