\chapter{FOBOS Analysis} \label{chap:dataAnalysis}
The Analysis module is used to perform DPA attack on power traces collected by the Acquisition module. In this section we provide description of the configuration and
usage of this module. To perform analysis, to inputs are needed, the power traces and and hypothetical power model.
\section{Power Model}
Power model or hypothetical power data is user provided. FOBOS uses a text file for each key byte.
This file includes a line for each key guess value (i.e. 0-255). Each line inculdes hypothetical power value for the specific key guess for all encryptions. Each value is an integer and separated from next value by a space. Fig ~ \ref{fig:fobos-estpower} is an example for one key byte. The firs number is the estimated power when the byte equals zero for the frist encryption, first value in the second line is the estimated power when the key byte equals one for the first encryption and so on.

\begin{figure}[H]
\begin{Verbatim}[frame=single]
4 6 3 4 5 1 5 4 4 2 4 3 5 6 3 4 4 3 4 2 . . . .
3 2 7 6 5 6 5 5 7 3 5 4 4 2 6 5 2 3 2 5 . . . .
5 4 3 6 4 5 3 4 3 3 6 5 3 0 2 5 6 5 6 5 . . . .
. . . .
\end{Verbatim}
\caption{\label{fig:fobos-estpower}Hypothetical Power Model}
\end{figure}

FOBOS expects to find these files at \$fobos/data/. File names should be hypo-power-byte-BYTE\_NUMBER.txt.

\section{Analysis Configuration}
Here we list all the configuration parameter used in the FOBOS Analysis and descriton of usage:

\begin{table}[H]
  \begin{center}
    \caption{signalAlignmentParams.txt}
    \begin{tabular}{|p{4cm}|p{2cm}|p{7cm}|}
    	\hline
       Parameter & Possible Values  & Description \\ \hline
       CAPTURE\_MODE & MUL|SINGLE & Use SINGLE when each trace captured \newline by the oscilloscpoe 
       								contains one trace and MULTI when the traces contains multiple encryptions \\ \hline
       																						
    TRIGGER\_THRESHOLD  & Float (e.g. 1.0) & Floating point number represnting minimum voltage to indicate a valid trigger. \\ \hline
    
  \end{tabular}
  \end{center}
\end{table}


\begin{table}[H]
  \begin{center}
    \caption{samplesSpacesDisp.txt}
    \begin{tabular}{|p{4cm}|p{2cm}|p{7cm}|} \hline 
    Parameter & Possible Values  & Description \\ \hline   
    SAMPLE\_WINDOW\_SIZE & Integer (e.g 2000) & Number of samples per trace to be used in analysis. \\ \hline
	SAMPLE\_WINDOW\_START & Integer (e.g 100) & The number of the first sample in the window. \\ \hline
    
    \end{tabular}
  \end{center}
\end{table}


\begin{table}[H]
  \begin{center}
    \caption{compressionParams.txt}
    \begin{tabular}{|p{4cm}|p{2cm}|p{7cm}|} \hline 
    Parameter & Possible Values  & Description \\ \hline
    COMPRESSION\_LENGTH & Integer (e.g. 10) & Number of samples to be compressed into one samples. \\ \hline
	COMPRESSION\_TYPE & MAX|MIN|MEAN & The operation to be performed to generated the compressed sample. \\ \hline
    \end{tabular}
  \end{center}
\end{table}

\begin{table}[H]
  \begin{center}
    \caption{traceExpunge.txt}
    \begin{tabular}{|p{4cm}|p{2cm}|p{7cm}|} \hline 
    Parameter & Possible Values  & Description \\ \hline   
    TRACE\_EXPUNGE\_PARAMS &  STD|VAR:BELOW:ABOVE|NO (e.g. VAR:0.0004:0.00045) &    Specifies the operation to be done on traces 
    																				and the lower/upper bounds of traces not to
    																				discard. \\ \hline
    \end{tabular}
  \end{center}
\end{table}


\begin{table}[H]
  \begin{center}
    \caption{postProcessesParams.txt}
    \begin{tabular}{|p{4cm}|p{2cm}|p{7cm}|} \hline 
    Parameter & Possible Values  & Description \\ \hline   
    SAMPLE\_SPACE\_DISPOSITION  &   &     \\ \hline
    COMPRESS\_DATA  & 1-3|NO  &     \\ \hline
    TRACE\_EXPUNGE  & 1-3|NO  &     \\ \hline
    \end{tabular}
  \end{center}
\end{table}

\begin{table}[H]
  \begin{center}
    \caption{projectPath.txt}
    \begin{tabular}{|p{4cm}|p{2cm}|p{7cm}|} \hline 
    Parameter & Possible Values  & Description \\ \hline   
    N/A & A director path (e.g. /fobos/workspace/test/1-test) & This file contains the path to the directory that contains data to be analysed.
    															If this file is missing, FOBOS will prompt user to select the directory and save user
    															select to this file. \\ \hline
    
    \end{tabular}
  \end{center}
\end{table}

\section{Running Data Analysis}

Data Analysis can be run as follows: \newline
\texttt{cd \$fobos} \newline
\texttt{python dataAnalysis.py} \newline

Once this is done, the script reads the measured and hypothetical power data, runs DPA and produces several output files.
The script prompts the user for the directory that contains the traces (the \$attmpt dirctory) and then uses the power data in \$attempt/Measurements as input. Once user specifies the directory it is save in \$fobos/config/projectPath.txt and used in subsequent runs without prompting the user. To make FOBOS ask for the \$attempt directory, remove this file.
The script will create a new directory each time it runs. This directory is created in the project directory.
Output files will be save in \texttt{\$fobos/\$workspace/\$project/\$attempt/analysis/\$analysis-attempt/}.
Where \$analysis-attempt is a folder with a unique name created each time the script runs.
