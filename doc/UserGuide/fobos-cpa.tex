\chapter{Correlation Power Analysis} \label{chap:fobos-cpa}

Once acquisition is complete, Correlation Power Analysis CPA can be performed.
User must provide their own power model and calculate hypothetical power for each key guess.
FOBOS takes the hypothetical power for each key guess and use a correlation method e.g. Pearson’s coefficient and calculates the correlation values.
The key guess that achieves the highest correlation is a candidate for correct key.

The Analysis module is used to perform DPA attack on power traces collected by the Acquisition
module. To Perform CPA, two inputs are needed, the power traces and and hypothetical power.


\section{Steps to perform CPA using FOBOS}
\subsection{Hypothetical power calculation}
Note: This description uses an example case where key is guessed one byte at a time. \newline
Power model or hypothetical power data is user provided. FOBOS uses a text file for each key
byte. This file includes a line for each key guess value (i.e. 0-255). Each line inculdes hypothetical
power value for the specific key guess for all encryptions. Each value is an integer and separated
from the next value by a space. The text below is an example for one key byte. The first number is the
hypthetical power when the byte equals zero for the frist encryption, first value in the second line is
the estimated power when the key byte equals one for the first encryption and so on.
FOBOS expects to find these files at \texttt{\$fobos/data/}. File names should be HPower\_byte\_$<$BYTE NUMBER$>$.txt.

Here is sample of \texttt{Hpower\_byte\_0.txt}

\begin{verbatim}
6 4 4 2 5 4 3 5 7 4 5 5 7 3 4 6 5 2 4 5 3 4 3 4 7 4 ….
5 3 4 5 4 2 5 7 4 4 2 4 3 2 4 4 3 4 2 4 3 6 3 2 1 5 ….
.
.
.
\end{verbatim}

\section{CPA configuration}

There are few configuration files that controls the CPA analysis.
Here we list all the configuration parameters used in the FOBOS Analysis and description of their usage.
\subsection{Data Analysis Parameters}

File: \texttt{fobos/config/dataAnalysisParams.txt}
\begin{itemize}
 \item WORK\_DIR \newline
 Directory to save analysis files(Inside the measurements directory). \newline
 Possible Values: file name \newline
 Example: analysis 
 \item MEASUREMENT\_WORK\_DIR \newline
 The name of the measurement directory. \newline
 Possible Values: directory name. \newline
 Example: workspace
 \item TAG \newline
 The type of prefix for the directory name. Used to distinguish different runs. \newline
 Possible Values: counter \newline
 Example: counter
\end{itemize}

\subsection{Sample Space Disposition}

File: samplesSpacesDisp.txt
\begin{itemize}
 \item SAMPLE\_WINDOW\_SIZE \newline
  Number of samples (per trace) to be used in analysis. \newline
  Possible values : Number (e.g. 2000)
 \item SAMPLE\_WINDOW\_START \newline
 The number of the first sample in the window.\newline
 Possible values: Number (e.g. 100)
\end{itemize}

\subsection{Compression Parameters}
File: signalAlignmentParams.txt
\begin{itemize}
 \item COMPRESSION\_LENGTH \newline
 Number of samples to be compressed into one samples. \newline 
 Possible values : Number (e.g. 10)
 \item COMPRESSION\_TYPE \newline 
 The operation to be performed to generate the compressed sample.\newline
 Possible values : MEAN $|$ MAX $|$ MIN
\end{itemize}

\subsection{Sample Analysis Configuration Files}

\begin{verbatim}
dataAnalysisParams.txt 
####################################################
WORK_DIR = FOBOSAnalysis 
MEASUREMENT_WORK_DIR = FOBOSWorkspace 
TAG = counter


compressionParams.txt 
#################################################### 
######## Compression Module Parameters ############# 
#################################################### 
COMPRESSION_LENGTH = 10 
COMPRESSION_TYPE = MEAN # MAX|MIN|MEAN 

sampleSpaceDispParams.txt 
#################################################### 
#### Sample Space Disposition Module Parameters #### 
#################################################### 
SAMPLE_WINDOW_SIZE = 3000 
SAMPLE_WINDOW_START = 3000

\end{verbatim}


\section{Running Data Analysis}

Data Analysis can be run as follows: \newline
\texttt{\$cd \$fobos} \newline
\texttt{\$python dataAnalysis.py} \newline

Once this is done, the script reads the measured and hypothetical power data, runs CPA and
produces several output files. The script prompts the user for the directory that contains the traces
and then uses the power data in the measurements directory as input.  The script
will create a new directory each time it runs. This directory is created in the project directory. Output files will be save in \texttt{fobos/$<$workspace$>$/$<$projectDir$>$/analysis/}.
