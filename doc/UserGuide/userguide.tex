% This is LLNCS.DOC the documentation file of
% the LaTeX2e class from Springer-Verlag
% for Lecture Notes in Computer Science, version 2.4
\documentclass{llncs}
\usepackage{llncsdoc}
\usepackage{setspace}
\usepackage{cite,amsmath,amssymb,mathrsfs,amsxtra,stmaryrd} %% math symbols
\usepackage{indentfirst}    %% This idents the new paragraphs automatically
%\usepackage{harvard}        %% Harvard citation style.
\usepackage[refpage]{nomencl} %% This allows to create a nomenclature page
\usepackage{multirow, xcolor, colortbl}
\usepackage{rotating}
\usepackage{array}
\usepackage{textcomp}
\usepackage{epsfig,cite,amsfonts,subfigure,graphicx } % use this in with other package above in the title
\usepackage{textcomp}
\usepackage{eurosym}
\usepackage{algorithmic}
\usepackage{algorithm}
\numberwithin{algorithm}{chapter}
%
\setlength{\arrayrulewidth}{0.5mm}
%\setlength{\tabcolsep}{18pt}
\renewcommand{\arraystretch}{1.5}
\definecolor{grey}{rgb}{190,190,190}
\begin{document}
%
\begin{flushleft}
\LARGE\bfseries Flexible Opensource BOard for
Sidechannel analysis 
\end{flushleft}
\rule{\textwidth}{1pt}
\vspace{2pt}
\begin{flushright}
\Huge
\begin{tabular}{@{}l}
FOBOS
{\Large Version 0.1}
\end{tabular}
\end{flushright}
\rule{\textwidth}{1pt}
\vfill
%
\newpage
\tableofcontents
\newpage
\section{FOBOS - Capture Module}
\section{FOBOS - Analysis Module}
FOBOS's analysis module uses a set of python scripts to post process the raw 
measurement data obtained from the oscilloscope and perform analysis on the obtained data
Various functions implemented in the Analysis module is described below:\newline

\begin{table}
\caption{Config Extract Functions}
\begin{tabular}{ |p{2cm}||p{11cm}|  }
 \hline
 \multicolumn{2}{|c|}{\cellcolor{teal}\textbf{configExtract.extractAnalysisConfigAttributes()}} \\
 \hline
 Usage & \texttt{\$configExtract.extractAnalysisConfigAttributes(filename)}\\ \hline
 Description & Loads the configuration attributes required for various analysis sub-modules\\ \hline
 Inputs & file-name \\ \hline
 Outputs & None\\ \hline
\end{tabular}
\end{table}

\begin{table}
\caption{Signal Alignment Functions}
\begin{tabular}{ |p{2cm}||p{11cm}|  }
 \hline
 \multicolumn{2}{|c|}{\cellcolor{teal}\textbf{signalAlignmentModule.getAlignedMeasuredPowerData()}} \\
 \hline
 Usage & \texttt{\$signalAlignmentModule.getAlignedMeasuredPowerData()}\\ \hline
 Inputs & None \\ \hline
 Outputs & An M x N numpy array matrix\\ \hline
 Description & Aligns all the raw measured data obtained from the oscilloscope with respect to
 trigger signal. This function returns a M x N numpy array matrix where there are M encryptions/decryptions 
 and N oscilloscope sample points per measurement\\ \hline
\end{tabular}
\end{table}

\begin{table}
\caption{Signal Alignment Functions}
\begin{tabular}{ |p{2cm}||p{11cm}|  }
 \hline
 \multicolumn{2}{|c|}{\cellcolor{teal}\textbf{signalAlignmentModule.acquireHypotheticalValues()}} \\
 \hline
 Usage & \texttt{\$signalAlignmentModule.acquireHypotheticalValues(filename)}\\ \hline
 Inputs & filename \\ \hline
 Outputs & An M x N numpy array matrix\\ \hline
 Description & Loads the hypothetical power model into an M x N numpy array where there are M 
 secret key guesses and N encryptions. This file is to be placed in 
 \verb|$fobos\powermodels| directory. \\ \hline
\end{tabular}
\end{table}
\end{document}
 
