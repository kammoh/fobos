\chapter{FOBOS Data Acquisition} \label{chap:dataAcquisition}

\section{FOBOS Acquisition}
Data Acquisition module is used to run encryptions on FOBOS hardware and collect traces. Plain text and key can be randomly generated or read from file. The output from this phase is a Numpy array file (.npy) that holds power traces.

\section{Requirements}
This module uses an osciloscpoe to measure and capture traces. Current version connects to oscilloscope using IP network.
The acqusition module runs on a Windows or Linux PC and connected to the FOBOS control board via USB cable.
FOBOS software and hardware need to be setup before Data Acqusition is run.

\section{FOBOS Acquisition Configuration}

Data Acquisition need to be configured. The configuration file is located at \texttt{\$fobos/config/dataAcquisitionConfig.txt}.
Table ~ \ref{acquisition-config} lists all configuration parameter needed to configure Data Acquisition.


\begin{table}[ht]
\begin{center}
\caption{\label{acquisition-config}acquisitionConfig.txt}
\begin{tabular}{|p{4cm}|p{2cm}|p{7cm}|} \hline 
Parameter & Possible Values  & Description \\ \hline   
Global Settings & & \\ \hline
MEASUREMENT\_FORMAT & dat & \\ \hline
LOGGING & INFO|DEBUG & Specifies log detail level. DEBUG is more verbose \\ \hline
CONTROL\_BOARD & Nexys2|Nexys3 & Control Board model\\ \hline
VICTIM\_RESET & Integer (e.g 11) & Number of DUT clock cycles before resetting the DUT \\ \hline
TIME\_OUT & Integer (e.g. 50000) & If DUT is not returning data after the number of clock cycles specified, a time out signal is sent to the PC \\ \hline
TRIGGER\_WAIT\_CYCLES & Integer (e.g 1) &  Number of DUT clock cycles before sending trigger signal to oscilloscope \\ \hline
TRIGGER\_LENGTH\_CYCLES & Integer (e.g 1) & Number of DUT clock cycles the trigger signal remains high\\ \hline
PLAINTEXT\_GENERATION &  USER|RANDOM  & Specifies if plain text is read from file or randomly generated \\ \hline
DATA\_FILE & File name (e.g. plaintext.txt) & Specifies the plain text file name. File is saved in the sources directory \\ \hline
KEY\_GENERATION & USER|RANDOM & Specifies if the key is read from file or randomly generated \\ \hline 
KEY\_FILE &  File name (e.g keys.txt) & Specifies the key file name. File is saved in the sources directory \\ \hline
INPUT\_FORMAT  & hex & Specifies input data format. Currently, only hexadecimal is supported \\ \hline
OUTPUT\_FORMAT & hex & Specifies output data format. Currently, only hexadecimal is supported \\ \hline
NUMBER\_OF\_ENCRYPTIONS\_PER\_TRACE & Integer (e.g 1) & Specifies the number of encryptions in each trace collected from the oscilloscope\\ \hline
BLOCK\_SIZE & Integer (e.g 16) & Cipher block size in bytes \\ \hline
KEY\_SIZE & Integer (e.g 16) & Cipher key size in bytes \\ \hline
DUMMY\_RUN & YES|NO & \\ \hline
NUMBER\_OF\_TRACES & Integer (e.g. 2000) & Number of encryptions/traces to be done/collected \\ \hline
CAPTURE\_MODE & MULTI|SINGLE & Specifies single encryption or multiple per trace\\ \hline
TRIGGER\_THRESHOLD & Float (e.g 1.0)  & Minimum trigger signal voltage to be considered a valid trigger \\  \hline
OSCILLOSCOPE & AGILENT|OPENADC & Oscilloscope model \\ \hline
OSCILLOSCOPE\_IP & IP address (e.g 192.168.0.10) & Oscilloscope IP address \\ \hline
OSCILLOSCOPE\_PORT & port number (e.g 5025) & Osciloscope port number \\ \hline
AUTOSCALE & YES|NO &  \\ \hline    
IMPEDANCE & FIFTY|ONEMEG & Oscilloscope impedance \\ \hline
VOLTAGE AND TIME RANGE OPTIONS & & \\ \hline       
CHANNEL1\_RANGE & ON|OFF|voltage range & Channel 1 vlotage range (full page) \\ \hline
CHANNEL2\_RANGE & ON|OFF|voltage range & Channel 2 vlotage range (full page) \\ \hline
CHANNEL3\_RANGE & ON|OFF|voltage range & Channel 3 vlotage range (full page) \\ \hline
CHANNEL4\_RANGE & ON|OFF|voltage range & Channel 4 vlotage range (full page) \\ \hline
TIME\_RANGE & Float (e.g 0.000028) & Trace time range (full page) \\ \hline
TIMEBASE\_REF & LEFT & \\ \hline    
TRIGGER\_SOURCE & & Cahnnel name (e.g CHANNEL2) &  Channed connect to trigger signal\\ \hline
TRIGGER\_MODE &  EDGE & Oscilloscope trigger mode \\ \hline   
TRIGGER\_SWEEP & NORM & \\ \hline
TRIGGER\_LEVEL & (e.g 1) & \\ \hline
TRIGGER\_SLOPE & POSITIVE & \\ \hline
ACQUIRE OPTIONS & & \\hline
ACQUIRE\_TYPE & NORM|PEAK|HRES|AVER & \\ \hline
ACQUIRE\_MODE & RTIM|ETIM|SEG & \\ \hline
\end{tabular}
\end{center}
\end{table}

\section{Running Data Acquisition}

Data Acquisition can be run as follows: \newline
\texttt{cd \$fobos} \newline
\texttt{python dataAcquisition.py} \newline
Once this is done, the script will run encryptions on hardware, collect traces from the oscilloscope and save traces into a file.
The script will create a new directory each time it runs. This directory is created in the project directory.
Traces will be save in \newline \texttt{\$fobos/\$workspace/\$project/\$attempt/Measurements/rawDataAligned.npy}.
Where \$attempt is a folder with a unique name created each time the script runs.
