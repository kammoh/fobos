\chapter{FOBOS Overview}
\section{Introduction}
FOBOS is a side-channel analysis tool that is flexible, open-source and includes tools needed for power data acqusition and analysis.
In this section we describe how to download and install FOBOS and describe the general procedure for using it to perform DPA attacks.

\section{Requirements}

\subsection{Linux Requirements}
The following items are requirements to running FOBOS on Linux machines:
\begin{enumerate}
\item Python need to be installed on the system. Installer can be downloaded from https://www.python.org.
\item Network configuration must allow the PC to be able to communicate to the Oscilloscope via IP network.
\end{enumerate}

\subsection{Windows Requirements}
The following items are requirements to running FOBOS on Windows machines:
\begin{enumerate}
\item Python need to be installed on the system. Installer can be downloaded from https://www.python.org/.
\item Network configuration must allow the PC to be able to communicate to the Oscilloscope via IP network.
\end{enumerate}


\section{Installation}
To install FOBOS software follow these steps:
\begin{enumerate}
\item Download FOBOS from https://cryptography.gmu.edu/fobos/.
\item Extract the file as follows: \newline
\texttt{tar xvzf fobos.tar.gz}
\end{enumerate}

\clearpage
\section{File Structure}
After FOBOS is extracted from the compressed file, its file structure looks like the following:

\tikzstyle{every node}=[thick,anchor=west, rounded corners, font={\scriptsize\ttfamily}, inner sep=2.5pt]
\tikzstyle{selected}=[draw=blue,fill=red!10]
\tikzstyle{root}=[selected, fill=blue!30]
%underscore should be written as "\_"
%No empty lines allowed
\begin{tikzpicture}[%
    scale=.7,
    grow via three points={one child at (0.5,-0.65) and
    two children at (0.5,-0.65) and (0.5,-1.1)},
    edge from parent path={(\tikzparentnode.south) |- (\tikzchildnode.west)}]
  \node [root] {fobos}
    child { node [selected] {bin}
        child { node [selected] {analysis}
                child { node {\texttt{\underline{{ }{ }}init\underline{{ }{ }}.py}}}
                child { node {plottingModule.py}}
                child { node {postProcessingModule.py}}
                child { node {sca.py}}
                child { node {signalAnalysisModule.py}}
                child { node {statisticsModule.py}}
        }
        child { node at (10.5,-3.0) [selected] {globals}
                child { node {\texttt{\underline{{ }{ }}init\underline{{ }{ }}.py}}}
                child { node {cfg.py}}
                child { node {configExtract.py}}
                child { node {dataGenerator.py}}
                child { node {globals.py}}
                child { node {oscilloscopeFunctions.py}}
                child { node {printFunctions.py}}
        }
         child { node at (5.0,-3.5) [selected] {oscilloscope}
                 child { node {\texttt{\underline{{ }{ }}init\underline{{ }{ }}.py}}}
                 child { node {agilent\-socket.py}}
                 child { node {oscilloscope\_agilent.py}}
                 child { node {oscilloscope\_core.py}}
                 child { node {oscilloscope\_global.py}}
                 child { node {oscilloscope\_sockets.py}}
                 child { node {oscilloscope\_visa.py}}
                 child { node {viewPlots.py}}
                 child { node at (6.0,0.0)[selected]{testing}
                     child { node {channel1.dat}}
                     child { node {channel1.png}}
                     child { node {oscilloscope\_config.txt}}
                     child { node {oscilloscope\_core.py}}
                     child { node {oscilloscope\_sockets.py}}
                     child { node {preambleChannel1.dat}}
                     child { node {rawDataProcess.py}}
                }
         }
         child { node at (0,-4.0) [selected] {usb}
                 child { node {\texttt{\underline{{ }{ }}init\underline{{ }{ }}.py}}}
                 child { node {\texttt{\underline{{ }{ }}init\underline{{ }{ }}.pyc}}}
                 child { node {usbcomm\_core.py}}
                 child { node {usbcomm\_core.pyc}}
                 child { node {usbcomm\_global.py}}
                 child { node {usbcomm\_global.pyc}}
                 child { node [selected]{test}
                     child { node {contains the code for testing the usb setup}}
         }
                 }
         child { node at (0,-7.9) {.dataAnalysis.py}}
         child { node at (0,-7.9) {.dataAcquisition.py}}
    }
    child { node at (9.0,-10.7) [selected] {config}
        child { node {aquisitionConfig.txt}}
        child { node {compressionParams.txt}}
        child { node {config.txt}}
        child { node {dataAnalysisParams.txt}}
        child { node {postProcessesParams.txt}}
        child { node {projectPath.txt}}
        child { node {sampleSpaceDispParams.txt}}
        child { node {signalAlignmentParams.txt}}
        child { node {traceExpungeParams.txt}}
    }
    child { node at (0,-10.8) [selected] {data}
        child { node {connectiontest.txt}}
        child { node {keys.txt}}
        child { node {plaintext.txt}}
        child { node {others files}}
    }
    child { node at (0,-13.0) [selected] {doc}
        child { node at (3.0,0.0)[selected] {PmodConnector}
            child { node {pmod\_connector.pdf}}
            child { node {pmod\_connector.tex}}
        }
        child { node at (0.0,0.0)[selected] {UserGudie}
            child { node {Userguide.pdf}}
            child { node {pmod\_connector.tex}}
        }    
    }
    child { node at (0,-15.0) [selected] {sources}
        child { node at (4.0,0.0)[selected] {bridgeconnector}
            child { node {pmod\_connector.pdf}}
            child { node {pmod\_connector.tex}}
        }
        child { node at (0.0,0.0)[selected] {vhdl}
            child { node {Userguide.pdf}}
            child { node {pmod\_connector.tex}}
        }
        child { node {pmod\_connector.tex}}
    }
    child { node at (0,-20) {\dots}};
\end{tikzpicture}


\section{Hardware Setup}
FOBOS hardware consist of the following components:
\begin{enumerate}
\item FOBOS Control board.
\item FOBOS Victim board a.k.a DUT.
\item Control PC.
\item Power trace capturing device (e.g oscilloscope).
\end{enumerate}

FOBOS control and victim boards are standard FPGA boards that need to be configured. FOBOS control board harware description is provided along with a victim wrapper. Victim cipher is user provided.
FPGA boards need to be connected to other components. For details, please refer to Chapter ~\ref{chap:hardware-config}.

\section{Trace Acquisition}
After FOBOS software and hardware are setup, encryptions can be run on the victim board and traces collected.
This is refered to as Data Acquisition.
To perform data acquisiton, two steps need to be done:
\begin{enumerate}
\item Edit the configuration file \texttt{\$fobos/cofig/acquisitionConfig.txt}. Configuration parameters description provided in Chapter ~ \ref{chap:dataAcquisition}.
\item Run the dataAcquistion script as follows: \newline
\texttt{cd \$fobos} \newline
\texttt{python dataAcquisition.py}
\end{enumerate}
 
\section{Data Analysis}
Running DPA attack on the traces collected in the Data Aquisition phase is refered to as Data Analysis.
Data Analysis uses two inputs; measured power traces and hypothetical power traces.
To perform Data Analysis, Three steps need to be done:
\begin{enumerate}
\item Generate hypothetical power traces.
\item Edit configuration files at \$fobos/config. Details on configuration parameters provided in Chapter ~ \ref{chap:dataAnalysis}.
\item Run the the dataAnalysis script as follows: \newline
\texttt{cd \$fobos} \newline
\texttt{python dataAnalysis.py}
\end{enumerate}